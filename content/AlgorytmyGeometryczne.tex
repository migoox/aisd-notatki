\subsection{Zasada prawej ręki}

\subsection{Sprawdzenie czy dwa odcinki się przecinają}
Przypuśćmy, że dysponujemy przestrzenią $R^3$ oraz odcinkami danymi przez pary punktów 
$(P_1, P_2)$ oraz $(P_3, P_4)$. Naszym celem będzie sprawdzenie
czy istnieje punkt wspólny pomiędzy tymi odcinkami.

Na początku omówmy geometryczną interpretację iloczynów skalarnych, którą wykorzystamy w dalszych rozważaniach.

Dane są wektory $\vec{u},\vec{v} \in R^2$ oraz prosta $L$ prostopadła
do $\vec{u}$ przechodząca przez punkt $(0,0)$. Wtedy iloczyn skalarny 
standardowy $\vec{u} \cdot \vec{v}$
jest 
\begin{itemize}[]
	\item dodatni jeśli groty $\vec{u}$ oraz $\vec{v}$ są po tej samej stronie płaszczyzny podzielonej przez prostą $L$,
	\item ujemny jeśli groty $\vec{u}$ oraz $\vec{v}$ są po  przeciwnych stronach płaszczyzny podzielonej przez prostą $L$,
	\item zerowy w przypadku kiedy wektor $\vec{v}$
	leży na prostej $L$.
\end{itemize}

Teraz możemy sformuować uogólnienie zasady prawej ręki.
Dane są wektory $\vec{u},\vec{v},\vec{w} \in R^3$.
Wtedy jeśli $(\vec{u} \times \vec{v})
\cdot (\vec{u} \times \vec{w}) > 0$, to bazy
$(\vec{u}, \vec{v},  (\vec{u} \times \vec{v}))$ oraz
$(\vec{u}, \vec{w},  (\vec{u} \times \vec{w}))$ są zgodne
co interpretujemy geometrycznie, jako sytuację, w której 
groty wektorów 
$\vec{v}, \vec{w}$ są po tej samej stronie względem prostej,
na której leży wektor $\vec{u}$.


Pomysł na algorytm polega na sprawdzeniu czy punkty $P_1$
oraz $P_2$ znajdują po przeciwnych stronach prostej 
przechodzącej przez punkty $P_3$ i $P_4$ oraz 
na sprawdzeniu czy punkty $P_3$
oraz $P_4$ znajdują po przeciwnych stronach prostej 
przechodzącej przez punkty $P_1$ i $P_2$. Jeśli odpowiedź
na tak postawione pytanie jest twierdząca, to zadane
odcinki muszą się przecinać.


\begin{algorithm}[H]
	\caption{Sprawdzenie, czy dwa odcinki się przecinają w przestrzeni $R^3$}
	\begin{algorithmic}[1]
		\Procedure{SegmentIntersectionR3($P_1$, $P_2$, $P_3$, $P_4$)}{}
		\If{$P_1 = P_3$ lub $P_1 = P_4$ lub $P_2 = P_3$ lub $P_2 = P_4$}
		\State \Return \textit{true}
		\EndIf
		\State $\vec{v}_{12} \gets P_2 - P_1$
		\State $\vec{v}_{34} \gets P_4 - P_3$
		\State $\vec{w} \gets \vec{v}_{12} \times \vec{v}_{34}$	
		\If{$\vec{w} = \mathbf{0}$} \Comment Jeśli odcinki są współliniowe
		\State $d \gets \max\{|P_1P_3|, |P_1P_4|, |P_2P_3|, |P_2P_4|\}$
		\If{$d \leq |P_1P_2| + |P_3P_4|$}
		\State \Return \textit{true}
		\EndIf
		\EndIf
		\State $\vec{v}_{13} \gets P_3 - P_1$
		\State $\vec{v}_{14} \gets P_4 - P_1$
		\If{$(\vec{v}_{12} \times \vec{v}_{14}) \cdot \vec{w}$ oraz 
		$(\vec{v}_{12} \times \vec{v}_{13}) \cdot \vec{w}$ są tego samego znaku}
		\State \Return \textit{false}
		\EndIf
		\State $\vec{v}_{32} \gets P_2 - P_3$
		\State $\vec{v}_{31} \gets P_1 - P_3$
		\If{$(\vec{v}_{34} \times \vec{v}_{31}) \cdot \vec{w}$ oraz 
		$(\vec{v}_{34} \times \vec{v}_{32}) \cdot \vec{w}$ są tego samego znaku}
		\State \Return \textit{false}
		\EndIf
		\State \Return \textit{true}
		\EndProcedure
	\end{algorithmic}
	\label{segment_intersection_r3}
\end{algorithm}

Zauważmy, że gdybyśmy powyższy problem rozwiązywali w przestrzeni $R^2$,
to podczas badania iloczynów wektorowych, jedyne co nas by interesowało
to znak 3-ej współrzędnej. Z zasady prawej ręki, jeśli
trzecia współrzędna wektora 
$\vec{u} \times \vec{v}$ jest dodatnia, to wektor $\vec{v}$
znajduje się po lewej stronie wektora $\vec{u}$ oraz jeśli 
jest ujemna to wektor $\vec{v}$
znajduje się po prawej stronie wektora $\vec{u}$. 

Fakt ten znacznie upraszcza sprawdzanie orientacji wektorów iloczynem wektorowym.

Trzecią współrzędną iloczynu wektorowego $\vec{u} \times \vec{v}$ w $R^2$
obliczymy korzystając ze wzoru
\[c = \vec{u}_x \vec{v}_y - \vec{v}_x \vec{u}_y.\]



\begin{algorithm}[H]
	\caption{Sprawdzenie, czy dwa odcinki się przecinają w przestrzeni $R^2$}
	\begin{algorithmic}[1]
		\Procedure{Prod2($\vec{u}$,$\vec{v}$)}{}
		\State \Return $\vec{u}_x \vec{v}_y - \vec{v}_x \vec{u}_y$
		\EndProcedure
		\Procedure{SegmentIntersectionR2($P_1$, $P_2$, $P_3$, $P_4$)}{}
		\If{$P_1 = P_3$ lub $P_1 = P_4$ lub $P_2 = P_3$ lub $P_2 = P_4$}
		\State \Return \textit{true}
		\EndIf
		\State $\vec{v}_{12} \gets P_2 - P_1$
		\State $\vec{v}_{34} \gets P_4 - P_3$
		\If{Prod2$(\vec{v}_{12}$, $\vec{v}_{34}) = 0$} \Comment Jeśli odcinki są współliniowe
		\State $d \gets \max\{|P_1P_3|, |P_1P_4|, |P_2P_3|, |P_2P_4|\}$
		\If{$d \leq |P_1P_2| + |P_3P_4|$}
		\State \Return \textit{true}
		\EndIf
		\EndIf
		\State $\vec{v}_{13} \gets P_3 - P_1$
		\State $\vec{v}_{14} \gets P_4 - P_1$
		\If{$\text{Prod2}(\vec{v}_{12}$, $\vec{v}_{14})$ $\cdot$ $\text{Prod2}(\vec{v}_{12}$, $\vec{v}_{13}) > 0$}
		\State \Return \textit{false}
		\EndIf
		\State $\vec{v}_{32} \gets P_2 - P_3$
		\State $\vec{v}_{31} \gets P_1 - P_3$
		\If{$\text{Prod2}(\vec{v}_{34}$, $\vec{v}_{31})$ $\cdot$ $\text{Prod2}(\vec{v}_{34}$, $\vec{v}_{32}) > 0$}
		\State \Return \textit{false}
		\EndIf
		\State \Return \textit{true}
		\EndProcedure
	\end{algorithmic}
	\label{segment_intersection_r2}
\end{algorithm}


\subsection{Problem przynależności punktu do wielokąta}


\subsection{,,Metoda zamiatania''}
\subsubsection{Znajdowanie długości sumy odcinków}
Mamy zbiór być może nakładających się odcinków w jednowymiarowej przestrzeni Euklidesowej, które siłą rzeczy
znajdują się na jednej prostej. Naszym zadaniem jest policzyć długość tych odcinków, ale w ten sposób, że jeśli jakiś
fragment przestrzeni należy do więcej niż jednego odcinka, to i tak liczymy go tylko raz (czyli liczymy pole sumy
teoriomnogościowej tych odcinków).

Aby rozwiązać ten problem skorzystamy z metody zamiatania. Dla
uproszczenia przyjmijmy, że odcinki z wejściowego zbioru $X$ są reprezentowane
jako dwuwymiarowe punkty, takie, że dla każdego punktu, $x$-owa
współrzędna jest taka sama.

W ogólności w przestrzeni mamy kilka serii
nakładających się odcinków (zanim jeden odcinek się skończy, to już się zaczyna inny). Gdy poznamy długość każdej
z takich serii, odpowiedzią będzie suma ich długości. Musimy wykryć, gdzie taka pojedyncza seria się zaczyna, a gdzie
kończy, czyli mieć jej zakres. Zakres da nam długość.

Idea rozwiązania polega na przejściu miotłą od najmniejszego względem 
$y$-owej współrzędnej punktu w górę. Podczas przechodzenia
będziemy kontrolować liczbę odcinków, które są w kontakcie z miotłą.

Jako, że nie możliwym jest przechodzenie w sposób ciągły, w
metodzie zamiatania rozważamy tzw. \textit{zdarzenia}, które pozwalają
sprowadzić ten problem do problemu dyskretnego.

W naszym przypadku, każde natrafienie miotły na dowolny punkt
traktujemy jako zdarzenie. Jeśli dojdzie do takiego zdarzenia
zapiszemy współrzędną $y$ oraz informację czy punkt zaczyna, czy też
kończy jakiś odcinek.

W algorytmie \ref{VerticalSegmentsUnionLength} przyjmujemy, że
struktura Event składa się ze zmiennej Coord, która reprezentuje
współrzędną $y$ oraz zmiennej IsStartingPoint, która 
reprezentuje informację, czy punkt zaczyna odcinek.

\begin{algorithm}[H]
	\caption{Znajdowanie długości sumy odcinków}
	\begin{algorithmic}[1]
		\Procedure{VerticalSegmentsUnionLength}{$X$}
		\State Niech \textit{Events} to lista zdarzeń typu Event
		\For{$AB \in X$}
			\If{$A_y \leq B_y$}
				\State Events.PushBack($A$, \textit{true})
				\State Events.PushBack($B$, \textit{false})
			\Else
				\State Events.PushBack($A$, \textit{false})
				\State Events.PushBack($B$, \textit{true})
			\EndIf
		\EndFor
		\State Posortuj rosnąco Events względem zmiennej Coord
		\State Niech \textit{level} oznacza liczbę nałożonych na siebie odcinków w danym momencie
		\State Niech \textit{startPoint} oznacza początek ciągłej sumy odcinków
		\State Niech \textit{distance} oznacza aktualnie policzoną długość
		\For{event $\in$ Events} \Comment Ten zapis rozumiemy jako przejście od pierwszego elementu do ostatniego elementu na liście
			\If{event.IsStartingPoint}
				\If{level $= 0$}
					\State startPoint = event.Coord
				\EndIf
				\State level $\gets$ level $+$ 1
			\Else 
				\State level $\gets$ level $-$ 1
				\If{level $= 0$}
					\State distance = distance event.Coord - startPoint
				\EndIf
			\EndIf
		\EndFor
		\State \Return distance
		\EndProcedure
	\end{algorithmic}
	\label{VerticalSegmentsUnionLength}
\end{algorithm}

\subsubsection{Znajdowanie pola sumy prostokątów}
Wyobraź sobie prostokąty. Teraz wyobraź sobie pionową linię zamiatającą. Idzie ona od lewej do prawej. Tym
razem zatrzymuje się ona za każdym razem, gdy natrafi na pionowy bok prostokąta (każdy prostokąt ma dwa takie
boki). Takie zatrzymanie nazywamy zdarzeniem. 

W czasie zatrzymania obliczamy, jaka jest długość przecięcia wszystkich prostokątów z linią zamiatającą. Oznaczmy tę długość $D$. Nazwijmy ją wysokością. Teraz wiemy, że ta wysokość
nie zmieni się do następnego zdarzenia. Dlatego jeśli jedno zdarzenie zdarzyło się na współrzędnej $x_1$, a następne na
$x_2$, to pole prostokątów między tymi zdarzeniami wynosi $(x_2-x_1)\cdot D$. Suma takich pól jest szukanym polem.

W algorytmie \ref{VerticalSegmentsUnionLength} przyjmujemy, że
struktura Event składa się ze zmiennej Coord, która reprezentuje
współrzędną $y$ oraz zmiennej IsStartingPoint, która 
reprezentuje informację, czy punkt zaczyna odcinek. Ponadto 
struktura zawiera zmienną Idx, która przechowuje
indeks prostokąta w odpowiedniej tablicy.

Ponadto struktura Rectangle tworzona jest przez zmienne MinX, MinY, MaxX, MaxY.

\begin{algorithm}[H]
	\caption{Znajdowanie pola sumy prostokątów}
	\begin{algorithmic}[1]
		\Procedure{RectangleUnionArea}{rectangles -- tablica prostokątów, $n$ -- liczba prostokątów}
		\State Niech \textit{Events} to lista zdarzeń typu Event
		\For{$i \in 1,2 \ldots, n$}
		\State Events.PushBack(rectangles$[i]$.MinX, \textit{true}, $i$)
		\State Events.PushBack(rectangles$[i]$.MaxX, \textit{false}, $i$)
		\EndFor
		\State Posortuj rosnąco Events względem zmiennej Coord
		\State Niech \textit{currentSegments} oznacza 
		\State Niech \textit{startPoint} oznacza początek ciągłej sumy odcinków
		\State Niech \textit{area} oznacza aktualnie policzone pole
		
		\For{event $\in$ Events} \Comment Ten zapis rozumiemy jako przejście od pierwszego elementu do ostatniego elementu na liście
		\If{currentSegments.Count > 0}
		\State distance $\gets$  VerticalSegmentsUnionLength(currentSegments)
		\State area $\gets$ area $+$  distance $\cdot$ (event.Coord - startPoint)
		\EndIf
		\State startPoint = event.Coord
		\State $A = (0, \text{rectangles}[\text{event.Idx}].\text{MinY})$
		\State $B = (0, \text{rectangles}[\text{event.Idx}].\text{MaxY})$
		\If{event.IsStartingPoint}
		\State currentSegments.Add($AB$)
		\Else
		\State currentSegments.Remove($AB$)
		\EndIf
		\EndFor
		
		\State \Return area
		\EndProcedure
	\end{algorithmic}
	\label{RectangleUnionArea}
\end{algorithm}

\subsubsection{Sprawdzenie czy w zbiorze odcinków istnieją przecinające się odcinki}


